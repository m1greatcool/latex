\documentclass{article}
\usepackage[T2A]{fontenc}
\usepackage{amsmath,amssymb}
\usepackage[russian]{babel}
\usepackage[utf8] {inputenc}
\usepackage{fancyhdr}
\usepackage{soulutf8}
\usepackage{graphicx}
\usepackage{amsthm}
% 
\pagestyle{fancy}

\relpenalty=10000
\binoppenalty=10000

\fancyfoot{}
\fancyhead[RO,LE]{\thepage}
\fancyhead[LO]{\leftmark}
\fancyhead[RE]{\rightmark}


%\sodef\so{}{.1em}{1em plus1em}{.3em plus.05em minus.05em}
\newcommand{\eqdef}{\stackrel{\mathrm{def}}{=}}
\newcommand{\z}{\par\noindent\textbf{Определение: }}
\newcommand{\Or}{\ensuremath{\mathrm{O}_2(\mathbb{R})}{ }}
\newcommand{\So}{\ensuremath{\mathrm{SO}_2(\mathbb{R})}{ }}
\def\theo{\par\textbf{Теорема: } }


\newtheorem{Def}{Определение}[section]
\newtheorem{cory}{Следствие}
\newtheorem{Th}{Теорема}
\newtheorem{Lem}[Th]{Лемма}
\begin{document}
\parindent=1cm
\begin{titlepage}
\newpage

\begin{center}
САНКТ-ПЕТЕРБУРГСКИЙ ГОСУДАРСТВЕННЫЙ УНИВЕРСИТЕТ\\
\end{center}

\vspace{8em}

\begin{center}
{\scshape \Large Курсовая работа} \\ 
\end{center}

\vspace{2em}

\begin{center}
\textsc{{\Large Классификация групп плоских кристаллов }}
\end{center}

\vspace{17em}



\newbox{\lbox}
\savebox{\lbox}{\hbox{Камбалин Арсентий Владимирович}}
\newlength{\maxl}
\setlength{\maxl}{\wd\lbox}
\hfill\parbox{11cm}{
\hspace*{5cm}\hspace*{-3cm}Студент:\hfill\hbox to\maxl{Камбалин Арсентий Владимирови\hfill}\\
\hspace*{5cm}\hspace*{-3cm}Преподаватель:\hfill\hbox to\maxl{Волков Дмитрий Юрьевич}\\
\\
\hspace*{5cm}\hspace*{-3cm}Группа:\hfill\hbox to\maxl{15.Б04-мм}\\
}


\vspace{\fill}

\begin{center}
Санкт-Петербург \\2017
\end{center}

\end{titlepage}
\tableofcontents
\thispagestyle{empty}
\cleardoublepage
\section*{Введение}\label{sec:intro}
\addcontentsline{toc}{section}{\protect\numberline{}Введение}%
В данной работе речь пойдет о так называемых "группах симметрий плоскости"{ }или "кристаллографических группах плосокости". Происхождение названия связано с тем, что самые известные группы симметрий относятся к симметриям кристаллических структур полная классификация которых, была проведена независимо двумя математиками - Ф.~С.~Евграфовым и А.~М.~Шёнфлисом.
	
Цель данной работы научить читателя за минимальное количество времени классифицировать группы симметрий плоскости. От читателя требуется знание таких понятий как метрика, также знание теории групп на уровне базовых определений(хотя для неподготовленного читателя  в тексте может встретится пара мест,  которые придётся принять на веру) и достаточный уровень математической культуры. В тексте будут встречаться классические обозначения, например для множества действительных чисел - $\mathbb{R}$ и тому подобные, некоторые обозначения будут отдельно обговариваться.

\newpage
\section{Базовые определения и конструкции}\label{sec:main_defs}
Начнём с понятия изометрии. Пусть у нас есть $X,Y$ - метрические пространства с метриками $d_x,d_y$ соответственно. Тогда изометрией мы назовём такое отображение между $X$ и $Y$, которое сохраняет расстояние. Если формально, то
\begin{Def}
	\textsf{Изометрией} называется отображение 
	$\phi \mid X \mapsto Y$, такое что $\forall x,y \in X(d_x(x,y) = d_y(\phi(x),\phi(y)))$.
\end{Def}
Нас не будет интересовать общий случай изометрий, мы будем рассматривать только изометрии вида $\phi \mid \mathbb{R}^2 \to \mathbb{R}^2$, обозначим множество всех таких изометрий как Isom$(\mathbb{R}^2)$.
Легко видеть, что композиция изометрий - изометрия, обратное к изометрии - изометрия, следовательно, множество всех изометрий является группой относительно композиции, тогда для произвольного подмножества W симметрической группой назовём
\begin{center}
$Sym(W) = \{\phi \in $  Isom$(\mathbb{R}^2) \mid \phi(W) = W  \}$ 
\end{center}
\subsection{Основные типы изометрий плоскости}
В этой часте мы опишем основные типы изометрий.

\par\medskip \textbf{Параллельный перенос} \medskip \par

\noindent Параллельным переносом множества $W$ называется отображение, которое сдвигает каждую точку на фиксированный вектор.

\begin{figure}[h]
\centering
  \includegraphics[width=5cm]{tr}
  \caption{\footnotesize{Параллельный перенос ни на что не похожего множества $W$ на вектор $v$}}
  \label{fig:fig1}
\end{figure}

\noindent То есть вектору $v$ ставим в соответствие отображение переноса $T_v$. Отображение  $T_v$ будет изометрией {(Почему?)}. Очевидны следующие свойства параллельного переноса:
\\
1)  $T_v \circ T_w = T_{v+w}$ \\ 
2) $T_{\alpha v} = \alpha T_v$ \\ 
3) $\forall v \exists w$ ($ T_v \circ T_w = T_0$) \smallskip

\noindent Последнее свойство избыточно(Почему?). Исходя из свойств следует что множество всех параллельных переносов $\mathbb{T}$ является, вообще говоря, подгруппой Isom$(\mathbb{R}^2)$ и это нам понадобится в дальнейшем.

\par\medskip \textbf{Отражение} \medskip \par
\noindent Следующим типом изометрии является отражение относительно прямой $l$, называемой осью отражения, проходящей через начало координат (если что, начало координат можно перенести соответствующим параллельным переносом на нужную прямую). Прохождение оси отражения через начало координат даёт линейность $f$, попробуйте сами убедится в том, что если радиус-векторы точек откладывать не от точки принадлежащей оси, то тогда отражение $f$ будет не линейно. 
Выпишем явно формулу отражения, точка всегда мыслится как радиус вектор.

\begin{figure}[h]
\centering
  \includegraphics[width=6cm]{ref}
  \caption{\footnotesize{Множество $W$ смотрит в зеркало }}
  \label{fig:fig1}
\end{figure}
Пусть $x$ точка, тогда точка $R(x)$ - это отражение $x$ относительно прямой $l$ с направляющим вектором $v$, тогда \smallskip
\par
$R(x) = 2\frac{<x,v>}{<v,v>}v - x$\\
Отметим, что $l$ прохохдит через начало координат, а формулу можно получить рассмотрев соответствующие проекции.
И это тоже изометрия(Почему?).
\par\medskip \textbf{Поворот} \medskip \par
 \noindent Пусть $\theta$ - угол на который мы хотим повернуть вектор $v$, тогда вспомним то, что в $\mathbb{R}^2$ поворот осуществляется с помощью домножение вектора на матрицу вида 
$$r = \begin{pmatrix}
cos(\theta) & -sin(\theta) \\
sin(\theta) & cos(\theta)
\end{pmatrix}$$
Это также изометрия. Можно ещё заметить, что $T_v \circ r \circ T_{-v}$ также поворот. 
Точто так же как и параллельные переносы -  множество поворотов вокруг одной точки на один и тот же угол образуют группу. Будем выделять группы поворотов $C_n$, где угол поворота будет $\frac{2\pi}{n}$
\medskip

\begin{figure}[h]
\centering
  \includegraphics[width=6cm]{rot}
  \caption{\footnotesize{Поворот на угол $\frac{9}{8}\pi$}}
  \label{fig:fig1}
\end{figure}

\noindent Также выделим \textbf{скользящее отражение} как композицию отражения и параллельного переноса вдоль оси отражения.
\subsection{Вспомогательные леммы и утверждения}
Чисто геометрически очевидны следующие утверждения:
\begin{enumerate}
\item Пусть $R$ отражения с осью $l$  и $T_v$ - такой параллельный перенос, что $T_v(l) = l'$. Тогда $ T_{v}\circ R \circ T_{-v}$ отражение с осью $l'$.
\item Пусть $r$ поворот вокруг начала координат на угол $\phi$ и $v ={ }\stackrel{\longrightarrow}{OP}$, P - точка конца вектора, тогда $T_v \circ r \circ T_{-v}$ поворот вокруг P на угол $\phi$
\end{enumerate}
Нужно установить несколько свойств выше введённых изометрий.  	Отметим, что параллельный перенос не имеет точек $x$ вида $T_v(x) = x, v \neq 0$, то есть неподвижных точек, тогда как поворот имеет одну неподвижную точку, а отражение целую прямую. Рассмотрим различные свойства композиций изометрий.
\begin{Lem}
Пусть r поворот вокруг начала координат на угол $\gamma { }(\gamma \neq 0) $ и $v$ - вектор, тогда $T_v \circ r$ это поворот на угол $\gamma$ вокруг точки $-(r - \mathrm{id})^{-1}(v)$
\begin{proof}
Бездоказательно(Это будет доказано позже) скажем, что любая изометрия - это или отражение, или поворот, или параллельный перенос, или скользящее отражение. Тогда если  $T_v \circ r$ поворот, то есть всего одна точка удовлетворяющая уравнению $r(x) + v = x$, ясно $\mathrm{x}$ - искомое. Тогда $v=(\mathrm{id}-r)(x)$, следовательно $x = -(r - \mathrm{id})^{-1}(v)$. Обратное к $(\mathrm{id}-r)$ существует, так как матрица отображения 
$$
(\mathrm{id}-r)(v) = \begin{pmatrix}
1 - cos(\gamma) & sin(\gamma) \\
-sin(\gamma) & 1 - cos(\gamma)
\end{pmatrix}
\begin{pmatrix}
v_1 \\
v_2
\end{pmatrix}
$$
очевидно, обратима. Очевидно, что поворот осуществляется также на угол $\gamma$, так как отображения $r$ и $ T_v\circ r$ отличаются сдвигом на вектор $v$.
%Надо подумать нормально
%(Чтобы доказать формально, нужно в матричном виде найти $x = -(r - \mathrm{id})^{-1}(v)$ и найти $()(x)$)
%Думай
\end{proof}
\end{Lem}
Также верна 
\begin{Lem}
Пусть f(x) - отражение с осью $l$ проходящей через начало координат и пусть $v \in \mathbb{R}^2$ и ${g}  = T_v \circ f$ -скользящее отражение. Если вектор $v$ перпендикулярен $l$, то g - отражение с осью $l +  v/2$ верно и обратное.
Если же $v$ не перпендикулярен $l$, то $g^2$ - параллельный перенос на вектор $v + f(v)$ - вектор параллельный линии $l$.
\begin{proof}
Скользящее отражение является отражением, если сохраняет направляющий вектор оси(какой-то оси). Из этого: $f(w) + v = w$, тогда $v = w - f(w)$. Это влечёт
$$
f(v) = f(w - f(w)) = f(w) -f(f(w)) = f(w) - w = -v
$$
То есть v перпендикулярен оси $l$. И обратно, если $v$ перпендикулярен $l$, тогда $f(v) = -v$, тогда $g(v/2) = f(v/2) + v = -v/2 + v = v/2$. Следовательно($g$ сохраняет линию $l+v/2$, а любая изометрия, которая сохраняет линию - отражение) $g$ отражение.
Таким образом $g$ - не тривиальное(не отражение) скользящее отражение, если $v$ не перпендикулярен $l$. Рассмотрим теперь $g^2$. Если $x\in \mathbb{R}^2$, то
$$
g^2(x) = g(f(x) + v) = f(f(x) + v) + v = f(f(x))+f(v) + v = x +f(v)+v
$$
таким образом $g^2$ - это параллельный перенос на вектор $f(v)+v$
\end{proof}
\end{Lem}

\par\medskip \textbf{Симметрические группы фигур} \medskip \par

Рассмотрим пример конечных симметрических групп некоторых фигур. Очевидно поворот на $\pi/3$ изометрия, так же на рисунке выделены линии являющиеся осями отражения.

\begin{figure}[h]
\centering
  \includegraphics[width=6cm]{six}
  \caption{\footnotesize{Шестиугольник }}
  \label{fig:fig1}
\end{figure} 
 Таким образом симметрическая группа правильного  шестиугольника состоит из поворотов и отражений. Такую группу будем называть \texttt{Диедральной} и обозначать $D_n$, n - кол-во сторон многоугольника. Определим её формально.
\begin{Def}
Диедральной группой назовём группу $D_n$ порождающуюся двумя элементами $f,r \in D_n$, для которых выполняются следующие отношения: \medskip
\\ \medskip
$1){ } f*f = e$
\\ \medskip
$2){ } r^n = e$
\\ \medskip
$3){ } f^{-1}*r*f = r^{-1}$
\end{Def}
r - можно интерпретировать как поворот на угол $\frac{2\pi}{n}$, а $f$ как отражение вокруг любой из осей.
Очевидно, что для конечных фигур параллельный перенос не будет являться изометрией, чтобы определить группу симметрий для следующей картинки (картинка не полна, она продолжается узором на всю плоскость). Нужно ввести понятие целочисленной решётки.
\begin{figure}[h]
\centering
  \includegraphics[width=6cm]{tessel}
  \caption{\footnotesize{Шестиугольник }}
  \label{fig:fig1}
\end{figure} 
\begin{Def}
Целочисленной решёткой порождённой векторами  $t_1,t_2$ называется:
$$
R = \{z_1*t_1+z_2*t_2 \mid t_1,t_2 \in \mathbb{R}^2, z_1,z_2\in \mathbb{Z}\}
$$
\end{Def}
\subsection{Что дальше?}
В этом разделе будут отмечены те знания из теории групп, которые должны быть известны читателю, этот раздел можно смело пропускать, он рекомендательный. 
В дальнейшем изложении будут важны многие термины и понятия теории групп такие как: Нормальная подгруппа, Гомоморфизм, Ядро гомоморфизма, порядок группы, класс смежности, прямое произведение групп. Вышеперечисленные определения являются базовыми и их можно найти в любом нормальном учебнике  по алгебре. Так же будет использоваться прямое полупроизведение и немного $\mathbb{Z}$ - модули про это можно почитать в книге Винберга. 
\section{Описание структуры группы $\mathrm{Isom}(\mathbb{R}^2)$}
В этом разделе будет рассмотрена очень важная группа(важная для нашей основной цели) $\mathrm{Isom}(\mathbb{R}^2)$ - это группа изометрий плоскости, подобное изложение было бы приемлимо и для $\mathrm{Isom}(\mathbb{R}^n)$, но мы ограничимся двумерным случаем.
В рассмотрении будут важны две подгруппы $\mathrm{Isom}(\mathbb{R}^2)$, это множество всех параллельных переносов плоскости $\mathbb{T}$ и множество всех изометрий сохраняющих неподвижно точку, назовём его $\mathrm{O}_2(\mathbb{R})$.

Будем описывать $\mathbb{T}$ как векторное пространство. Тогда будут использованы привычныне обозначения, такие как норма вектора, угол между векторами и т.д.
Отметим, что $\|{u -v}\|^2 = \|{u}\|^2+\|{v}\|^2 - 2\|{u}\|\|{v}\|\cos{\theta} $, где $\theta$ - угол между $u$ и $v$ 
Выделим некоторые свойства изометрий по отношению к векторам.
\begin{Lem}
Если $g$ - изометрия $\mathbb{R}^2$, такая что $g(0) = 0$(читай, сохраняет начало координат), тогда
$$
\forall u,v\in\mathbb{R}^2(g(u)*g(v) = u*v)\wedge(\angle(u,v) =\angle(g(u),g(v)))
$$
\begin{proof}
Очевидно следует из отмеченного выше равенства и того, что $\|g(u)\| = \|u\|$
\end{proof}
\begin{Lem}
Если изометрия сохраняет начало координат, то она линейна.
\end{Lem}
\begin{proof}
Любой вектор раскладывается по ортноромированному базису, так как изометрия сохраняет углы и расстояния, то ортонормированный базис переходит в ортонормированный. Пусть $t_1,t_2$ ортнормированный базис базис, тогда $w_1 = g(t_1),w_2  = g(t_2)$ - тоже.
Любой вектор $v$ раскладывается по базису, а коэффициентами при базисных векторах будут проекции. 
\begin{center}
$ v = (v,t_1)*t_1+(v,t_2)*t_2$
\\ \medskip
$g(v) = (g(v),w_1)*w_1 + (g(v),w_2)*w_2$
\\ \medskip
\end{center}
Но по лемме 3: $(g(v),w_1) = (v,t_1)$ и $(g(v),w_2) = (v,t_2)$ тогда
\begin{center}
$g(v+u) = (v+u,t_1)*w_1 + (v+u,t_2)*w_2 = (v,w_1)*w_1+(v,w_2)*w_2 + (u,w_2)*w_1$
\\ \medskip
$+(u,w_2)*w_2 = g(v) + g(u)$
\end{center}
Аналогично для $g(\lambda v) = \lambda g(v)$, это не указание - это призыв.
\end{proof}
\end{Lem}
\begin{cory}
Пусть $f \in \mathrm{Isom}(\mathbb{R}^2)$, тогда $f(x) = g(x) + b$, для некоторой $g$ - линейной изометрии.
\end{cory}
\begin{proof}
Положим $b = f(0), g(x) = f(x)-b$, $g$ - изометрия как композиция изометрий и по предыдущему свойству линейна.
\end{proof}

Если представить изометрию $g$ в виде матрицы $G$: матрица будет состоять из векторов столбцов $Gt_1,Gt_2$, где $t_1,t_2$ - это базис. В таком представлении ясно видно, что $GG^T = E$, где $E$ - единичная, так как это эквивалентно перемножению $Gt_1,Gt_2$. 
Множество матриц с таким свойством назовём ортогональной группой мы обозначим \Or, а входящие в неё матрицы будем называть ортогональными.
\section{Описание \Or} 
\subsection{Связь \Or и $\mathrm{Isom}(\mathbb{R}^2)$}
Ортогональная группа важна как сама по себе, так и в приложениях к нашей основной задаче.
Следующая лемма устанавливает связь между \Or  и $\mathrm{Isom}(\mathbb{R}^2)$.
\begin{Lem}
Пусть $H$ подгруппа изометрий $\mathbb{R}^2$ которые сохраняют начало координат. Тогда $H \cong \Or $.
\end{Lem}
\begin{proof}
Определим отображение $\sigma \mid \Or \mapsto H$, так что матрице $A$ сопоставляется изометря $x \mapsto Ax$ или $\sigma(A)(x)=Ax$ имеем
$$
\sigma(AB)(x) = (AB)x = A(Bx) = \sigma(A)(Bx) = \sigma(A)(\sigma(B)(x)) = \sigma(A)\sigma(B)(x)
$$
Следовательно $\sigma$ - групповой гомоморфизм.

Непосредственная проверка инъективности (как обычно, доказываем, что ядро тривиально: в нашем случае состоит из единичной матрицы)
Сюръективность следует из  Леммы 4 (Наша изометрия сохраняет начало координат по условию, следовательно линеена)  и того, что матрицы линейных изометрий удовлетворяют тождеству $AA^T=E$, как было отмечено ранее.
\end{proof}
Заметье, что мы недавно доказали, что любая изометрия является композицией параллельного переноса и линейной изометрии. Далее для удобства будем обозначать $\mathrm{I} =\mathrm{Isom}(\mathbb{R}^2)$, а $\Or$ просто О тогда, как было отмечено: $\mathrm{I} = O\mathbb{T}$.
Сформулируем в виде леммы:
\begin{Lem}
Подгруппа параллельных переносов является нормальной подгруппой группы $\mathrm{I}$.
\end{Lem}
\begin{proof}
Чтобы доказать нормальность достаточно доказать что
 $r\mathbb{T}r^{-1} \subset \mathbb{T}$ (Следует из определения нормальной подгруппы можно посмотреть в "Алгебра" { }Ленга). То есть нужно доказать, что  если $T_v$ - параллельный перенос, а $r$ - линейная изометрия( элемент группы О), то $r\circ T_v \circ r^{-1}$ тоже параллельный перенос.
$$
r(T_v(r^{-1}(x))) = r(r^{-1}(x) + v) = r(r^{-1}(x)) + r(v) = x + r(v)
$$ 
Воспользовались линейностью отражения. Таким образом это перенос на вектор $r(v)$. Также очевидно, что $\mathbb{T}\cap \mathrm{O} = \mathrm{id}$, так как перенос не сохраняет ни одну точку.
\end{proof}
Вспоминая определение группового полупроизведения покажем, что $\mathrm{I} = \mathrm{T}\rtimes_\phi \mathrm{O}$, где $\phi$ - это ограничение внутренних автоморфизмов группы $\mathrm{O}$ на группу $\mathbb{T}$. Напомним внутренние автоморфизмы - это отображения вида $\mathcal{T} \mapsto h \mathcal{T} h^{-1}$.
Операция группы на полупроизведении $\mathrm{T}\rtimes_\phi \mathrm{O}$ даёт нам следующее выражение 
\begin{center}
$
(t_vh)(t_uh') = (t_v ht_uh^{-1})(hh') = (t_v\phi_h(t_u))(hh') = (t_vt_{h(u)})(hh') 
$ 
\smallskip \par
$ = (t_{v+h(u)})(hh') $
\medskip
\end{center}
Используя представление $\mathbb{T}$ как вектора в $\mathbb{R}^2$, так же представление элементов $\mathrm{O}$ как матриц , тогда используя отображение $\varphi \mid \mathrm{O} \mapsto \mathrm{Aut}(\mathbb{T})$, такое что 
{$\varphi(A)(v) = Av$,}
Тогда полупроизведение в таком представлении перепишется в виде:
$$
(A,v)(B,u) = (AB,v+Au).
$$
Тогда будем использовать обозначение для представление произвольной изометрии $t_v \circ f$, где $t_v$ параллельный перенос, а $f$ - линейная изометрия, в форме матриц и записывать так $(A,v)$, $A$ соответствующая $f$ матрица.
\subsection{Структура \Or}
В предыдущем разделе мы установили связь между группой изометрий и ортогональной группой, узнали как выглядит произвольная изометрия вплоть до линейной, теперь нужно изучить вопрос линейных изометрий. Как мы знаем - линейная изометрия представляется как ортогональная матрица матрица, то есть матрица с свойством $A*A^T = E$, тогда $$\det(A*A^{T}) = \det(A)*\det(A) = \det(E) = 1,$$ следовательно $\det(A)=\pm 1$. Отображая $\mathrm{O}$ на множество всех вещественных чисел не равных нулю отображением вида $A \mapsto \det(A)$, получим групповой гомоморфизм, ядром которого будет являться множество матриц у которых определитель равен 1 это множество будет нормальной подгруппой в $\mathrm{O}$, так как ядро гомоморфизма - нормальная подгруппа. Обозначим её как \So, эта  подгруппа называется специальная ортогональная группа. Таким образом 
$$
\So = \{A \in \Or \mid \det(A) = 1 \}
$$
Отметим, что индекс (множество всех смежных классов по подгруппе)  \So как подгруппы обозначается как: [\Or : \So] и равен 2(Почему?).
Пусть $A$ - матрица из \Or в стандартном базисе, тогда
$$
\begin{pmatrix}
a & b \\
c & d \\
\end{pmatrix}^T
\begin{pmatrix}
a & b \\
c & d \\
\end{pmatrix} = 
\begin{pmatrix}
a^2 + c^2 & ab + cd \\
ab + cd & b^2 + d^2 \\
\end{pmatrix}
=
\begin{pmatrix}
1 & 0 \\
0 & 1 \\
\end{pmatrix}
$$
Отсюда видно, что $a^2 + c^2 = 1$ и $b^2 + d^2 = 1$, следовательно существует угол $\theta$, такой что $a = \cos{\theta}, c = \sin{\theta}$. Из условия, что $ab + cd = 0$ следует $b = -\sin{\theta}, d = \cos{\theta}$ или $b = \sin{\theta}, d = -\cos{\theta}$. В первом случае определитель матрицы будет равен 1, а во втором -1.(Проверьте). Таким образом любой элемент \So выглядит как
$$
\begin{pmatrix}
\cos{\theta} & -\sin{\theta} \\
\sin{\theta} & \cos{\theta} \\
\end{pmatrix},
$$
для некоторого угла $\theta$. В итоге мы получили, что любой элемент \So является поворотом на определённый угол. Тогда как в случае, когда $A\notin\So$
$$
A = \begin{pmatrix}
\cos{\theta} & \sin{\theta} \\
\sin{\theta} & -\cos{\theta} \\
\end{pmatrix}
$$
Воспользовавшись формулой приведённой выше, получим, что это отражение относительно оси $y = (\mathrm{tg}{ \theta/2})x$. Таким образом элементы \Or - это отражения и повороты, причём повороты являются элементами \So. Так как любая изометрия - это композиция линейной изометрии и параллельного переноса, но сейчас мы выяснили, что линейная изометрия - это либо поворот, либо отражение отсюда немедленно следует.
\begin{cory}
Любая изометрия плоскости - это либо отражение, либо поворот, либо паралельный перенос, либо скользящая симметрия.
\end{cory}
Заметим, что если $r \in \So$ и $f \notin \So $, тогда $rf \notin \So$, таким образом $rf$ отражение. Тогда $(rf)^2 = 1$, следовательно $frf = r^{-1}$. А теперь вспомните определение Диедральной группы, то есть подгруппы \Or порождённые некоторым поворотом и некоторым отражением - это диедральные группы. Все эти рассуждения наводят на следующую теорему. 
\begin{Th}
Пусть $\mathrm{S}$ конечная подгруппа в \Or. Тогда $\mathrm{S}$ изоморфна или циклической группе порядка $n$, или диедральной группе порядка $2n$, $n \in \mathbb{N}$ 
\end{Th}
\begin{proof}
Пусть $N = \mathrm{S}\cap\So$, и так как индекс \So в \O равен 2. Индекс $N$ в $\mathrm{S}$ не больше 2, так как $S/N \cong G \subset (\Or/\So)$, тогда имеют место 2 случая. 

1) $[\mathrm{S}:N] = 1$, тогда  $\mathrm{S} \subset \So$ следовательно является циклической группой (докажите, что любая группа состоящая из поворотов на угол $2\pi/n$ является циклической группой порядка $n$)

2) $[\mathrm{S}:N] = 2$, тогда помимо поворотов в S содержится отражение, но как было отмечено ранее любая такая подгруппа $\mathrm{S}$ является диедральной и используя теорему Лагранжа о порядке группы, получаем, что порядок $\mathrm{S}$ равен $2n$.
\end{proof}
\section{Симметричное замощение плоскости}
Англоязычный термин ''wallpaper pattern'' мне не удалось хорошо перевести, поэтому вместо этого будет использован термин ''Симметричное замощение'', термин не самый удачный, но лучше я не придумал, так же в данной работе не хотелось использовать термин "группа симметрий плоских кристаллов" или "кристалло. Под симметричным замощением плоскости подразумевается такое замощение у когорого есть определённые симметрии, а именно сдвиг на некоторый вектор не изменит замощение, поворот на некоторый угол не не изменит замощение. Примером является рисунок 4.
\begin{figure}[h]
\centering
  \includegraphics[width=6cm]{tessel2}
  \caption{\footnotesize{Симметрическое замощение }}
  \label{fig:fig1}
\end{figure} 
Вспомним определение симметрической группы подгруппа параллельных переносов это группа $\mathrm{Sym}(W)\cap\mathbb{T}$. Тогда эта подгруппа попрождается некоторой целочисленной решёткой. Дадим формальное опредление симметрическому замощению.
\begin{Def}
Симметрическим замощением $W$ называется такое подмножество $\mathbb{R}^2$, если подгруппа параллельных переносов порождается некоторой целочисленной решёткой.   плосокй кристаллографической группой  $W$ назовём $\mathrm{Sym}(W)$.
\end{Def}
Далее группу симметрий замощения будем просто называть кристаллографической группой подразумевая, не приписывая слово ''плосая''. 
Дальнейшее изложение будет вестись подобно тому, что было ранее, а именно для $W$ - кристаллографическая группа, ясно, что подгруппа параллельных переносов $T$ нормальна в $W$(Это было отмечено ранее для группы изометрий), мы посмотрим какие возникают фактор-группы $W/T$, так же определим действие $W/T$ на $T$, научимся строить $W$ по $W/T$ и $T$. В итоге с помощью всего того, что было получено ранее нами будут классифицированы 17  плоских кристаллографических групп.
\newpage
\section{Точечная группа симметрии}
\subsection{Определения и главные свойства}
Пусть $W$ - кристаллографическая  группа с целочисленной решёткой $T$. В этом разделе мы опишем до изоморфизма всевозможные факторгруппы $W/T$. Напомним, что если $\varphi(x)$ - изометрия, то $\varphi(x) = A(x) + b$, для некоторой матрицы $A \in \Or$ и $b \in \mathbb{R}^2$, как было отмечено ранее.Упрощая обозначения мы будем писать $\varphi = (A,b)$, тогда композиция изометрий будет определяться как:
$$
(A,b)(C,d) = (AC,Ad+b).
$$
Для обратного отображения:
$$
(A,b)^{-1} = (A^{-1},-A^{-1}b)
$$
Например в данных обозначениях параллельный перенос на вектор $b$, будет $(E,b)$, а линейная изометрия $(A,0)$. Дадим следующее важное определение:
\begin{Def}
Пусть $W$ - кристаллографическая группа. Точечной группой симметрии $W_0$ группы $W$ назовём множество
$$
\{A\in \Or \mid \exists b \in \mathbb{R}^2((A,b)\in W)\}
$$
\end{Def}
\noindent (Проверьте аксиомы группы.)

Далее будем говорить просто ''точечная группа''. Ясно, что точечная группа является подгруппой в \Or тогда интуитивна следующая теорема:
\begin{Th}
Если $W$ это кристаллографическая группа с целочисленной решёткой $T$ и точечной группой симметрий $W_0$ тогда $W_0 \cong W/T$.
\end{Th}
\begin{proof}
Определим групповой гомоморфизм $\phi$ формулой, \\
$(A,b) \mapsto A$, тогда $\phi(W) = W_0$. Ограничим $\phi$ на $W$, тогда в ядре $\phi$ будет лежать $T$(Так как до этого, там лежала подгруппа всех параллельных переносов $\mathbb{T}$), тогда пользуясь тем, что $W/\ker{\phi}\cong \mathrm{Im}\phi$:  получим нужный изоморфизм.
\end{proof}
Ранее мы обговаривали то, что группа \Or действует на $\mathbb{T}$ посредством сопряжений, выпишем явно:
$$
(A,0)(E,t)(A,0)^{-1} = (E,At),
$$
здесь мы использовали $\mathbb{T} \cong \mathbb{R}^2$. Ограничим это действие на $W$, тогда получим, что если $A \in W_0$ и $b \in \mathbb{R}^2$, то $(A,b) \in  W$, тогда из того, что $W$ группа следует $(A,b)(E,t)(A,b)^{-1} = (E,At) \in W$, таким образом $At \in T$. Эти рассуждения позволят нам определить всевозможные точечные группы симметрий, которое только могут появляться в кристаллографической группе. Докажем вспомогательную лемму.
\begin{Lem}
Точечная группа симметрии $W_0$ кристаллографической группы $W$ - конечна.
\end{Lem}
\begin{proof}
Пусть ${t_1,t_2}$ базис для $T$, и пусть $C$ - окрудность с центром в начале координат, содержащая $t_1$ и $t_2$. Очевидно, что только конечное число элементов группы $T$ может содержаться в окружности(Почему?). Так как $W_0$ - это подгруппа в \Or она действует как группа перестановок  на $T$ в $C$ или просто переставляет элементы $T$ в окружности $C$. Внутри окружности может поместиться только конечное число образов при действии $W_0$(важны действия на базисные векторы), однако так как ${t_1,_2}$ - базис, то любой элемент в $W_0$ определяется действиями на ${t_1,t_2}$, таким образом кол-во образов конечно, следовательно группа конечна. (Окружность здесь нужна именно за тем, что группа $W_0$ - это группа изометрий и поэтому она сохраняет расстояния и поэтому за окружность достаточно большого радиуса не выйдут элементы на которых подействовали с помощью $W_0$, которые лежали при этом в этой окружности).
\end{proof}
Сейчас мы определим всевозможные появляющиеся как точечные группы симметрий, напомним, что $C_n$ - циклическая группа порядка $n$, $D_n$ - диедральная группа порядка $n$:
\begin{Th}
Пусть $W_0$ точечная группа симметрия $W$. Тогда $W_0$ изоморфна одной из следующих 10-ти групп:
$$
C_1,C_2,C_3,C_4,C_6, 
D_1,D_2,D_3,D_4,D_6 
$$
\end{Th}
\begin{proof}
Из теоремы 7 известно, что любая подгруппа \Or изоморфна либо диедральной, либо циклической, так как $W_0$ подгруппа в \Or осталось узнать порядки.  
Рассмотрим $N = W_0 \cap \So$ это циклическая группа порождённая поворотом на определённый угол $\phi =2\pi/n$, для некоторого $n\in \mathbb{N}$, порядок такой группы будет равен $n$. Представим элементы этой группы в разных базисах(как матрицу действий на базисные векторы), в классическом базисе это будет известная матрица поворота:
$$
\begin{pmatrix}
\cos{\theta} & -\sin{\theta} \\
\sin{\theta} & \cos{\theta} \\
\end{pmatrix}.
$$
С другой стороны пусть ${t_1,t_2}$ базис целочисленной решётки и так как $\forall r\in W_0, \\  r(T) = T$ и $T = \mathbb{Z}t_1 \bigoplus \mathbb{Z}t_2$, то матрица в данном базисе будет выглядет так:
$$
\begin{pmatrix}
a & b\\
c & d \\
\end{pmatrix}
$$
Где $a,b,c,d \in \mathbb{Z}$. Так как это две матрицы одной и той же изометрии, следовательно они сопряженые(матрицей перехода от одного базиса к другому). У сопряжённых матриц совпадают следы(сумма диагональных элементов). Таким образом $a + d = 2\cos{\phi}$, но так как $a$ и $d$ целые, то  $\cos{\phi} \in \{1/2,0,-1/2\}$. Ясно, что $\phi$ может принимать значения 
$$
\theta = \{\pi/3,\pi/2, 2\pi/3,\pi,2\pi \} = \{2\pi/6,2\pi/4,2\pi/3,2\pi/2,2\pi/1 \}
$$
Следовательно порядок $N$ равен $n \in \{1,2,3,4,6\}$. 
\end{proof} 
Далее мы докажем, что точечная группа симметрии полностью определяется её кристаллографической группой. Говоря более точно будет доказано, что если две кристаллографические группы изоморфны, то изоморфны их точечные группы симметрий. Докажем следующую важную лемму:
\begin{Lem}
Пусть $W,W'$ - кристаллографические группы с целочисленными решётками $T$ и $T'$ и множество 
$$
W_n = \{x\in W \mid xw^n=w^nx,{ } \forall w\in W\}
$$
Тогда \smallskip \\ 
1) $T = W_n$, тогда и только тогда когда $n$ кратно $[W:T]$ . \\
2) Если $W\stackrel{\varphi}{\cong}W',$ то $\varphi(T) = T'$
\end{Lem}
\begin{proof}
\par 1) Докажем, что $T \subset W_n$ \\
Пусть $n$ кратно $[W:T]$,  число $[W:T]$ совпадает с порядком $W/T \cong W_0$, тогда $\forall w \in W_0(w^n=e)$, следовательно $w^n\in T$, следовательно $xw^n=w^nx \in T$ как композиция параллельных переносов(строго: группа $T$ абелева) следовательно $T \subset W_n$ .\\
Докажем, что $W_n \subset T$ \\  
Предположим теперь $x \in W_n$. Тогда пусть $x = (A,b),b\in \mathbb{R}^2$. Рассмотрим $w = (E,t),t \in \mathbb{R}^2$. 
Тогда $w^n = (E,nt)$(по формуле, что приводилась выше). Так как $x \in W_n$ из определения $xw^nx^{-1} = w^{n}$. Заметим, что $xw^nx^{-1}$ это действие \Or на $\mathbb{T}$ посредством сопряжений, тогда верно(формула выше):
$$
xw^nx^{-1} = (E,A(nt))
$$
следовательно $A(nt) = nt,{ }\forall t \in \mathbb{R}^2$ , так как $A$  определяется через действие на векторы базиса, можно заключить, что если ${nt_1,nt_2}$ базис, а $A(nt_{1,2}) = (nt_{1,2})$, то $A = E$, следовательно $x = (E,b) \in T$ следовательно $W_n = T$
\par 2) Пусть $W \stackrel{\varphi}{\cong} W'$, очевидно(из свойства группового гоморфизма), что $\varphi(W_n) = W'_n, \forall n\in \mathbb{N}$. Пусть $m = [W:T]$, и $m' = [W':T']$, положим $n = mm'$, тогда из пердыдущего пункта следует, что $W'_n = T'$ и $W_n = T$, следовательно $\varphi(T) = T'$
\end{proof}
Докажите следующее следствие(профакторизуйте $W$ по $T$ потом воспользуйтесь предыдущей леммой)
\begin{cory}$W,W'$ кристаллографические группы, тогда из
$W \cong W'$, следует $W_0 \cong W'_0$
\end{cory}
На самом деле при более тщательном рассмотрении изоморфизма между кристаллографическими группами можно доказать более сильное утверждение. Далее мы получим необходимый критерий того, чтобы две точечные группы разных кристаллографических групп были изоморфны. С помощью этого критерия в совокупности с предыдущим следствием мы сможем доказывать, что две кристаллографические группы не изоморфны. 
\par Пусть $W$ кристаллографическая группа с целочисленной решёткой $T$ и точечной группой $W_0$, действие $W_0$ на $T$ влечёт гомоморфизм групп $W_0 \mapsto \mathrm{Aut}(T) \cong \mathrm{Aut}(\mathbb{Z}^2)$. Фиксируя базис ${t_1,t_2}$ в $T$, элементы $\mathrm{Aut}(\mathbb{Z}^2)$ могут быть представлены как матрицы $2\times2$, так как это группа изоморфизмов, все матрицы входящие в это представление обратимы, но так как элементами матриц являются матрицы с целыми коэффициентами, то при обращении матрицы важно, чтобы они оставались целыми, из формулы обращения матрицы будет следовать, что для любой обратимой матрицей над $\mathbb{Z}$ её определитель будет равен $\pm 1$(Простите за тавтологию).
\begin{Th}
Пусть $W\stackrel{\varphi}{\cong}W'$, $T,T'$ - соответствующие целочисленные решётки, $W_0,W'_0$ - соответствующие точечные группы. Фиксируя базисы для $T$ и $T'$ и элементы $W_0,W_0'$ в этих базисах(Представления элемента $W_0$ в виде матрицы). Отображение $\varphi$ ограниченное на $T$ будет линейным изоморфизмом с матрицей $U$, а индуцированный изоморфизм $W_0 \cong W'_0$ будет осуществляться отображением $A \mapsto UAU^{-1}$.
\end{Th}
\begin{proof}
Возьмём изоморфизм $\varphi \mid W \mapsto W'$. Из Леммы 11 пункта 2, ясно, что ограничение $\varphi$ на $T$ это изоморфизм $T$ на $T'$. Предположим что $\{t_1,t_2\}$ это базис в $T$ и $\{t'_1,t'_2\}$  базис для $T'$. Тогда мы имеем 
$$
\varphi(t_1) = at_1 + bt_2
$$
$$
\varphi(t_2) = ct'_1 + dt'_2 
$$
$a,b,c,d \in \mathbb{Z}$. Обозначим
$$U = \begin{pmatrix}
a & c \\
b & d \\
\end{pmatrix}$$
Матрица $U$ не вырождена(как матрица перехода от одного базиса к другому), так же у этой матрицы целые коэффициенты поскольку она является матрицей изоморфизма между целочисленными решётками и так как целые линейные комбинации не должны зависеть от базиса, то коэффициенты должны быть целыми.(Это предложение неявно отсылает читателя к определению и свойствам $\mathbb{Z}$ - модуля, посмотреть можно, например, в книжке Городенцева по алгебре.)
\\ Заметим, что так как $U$ матрица изоморфизма между  $\mathbb{Z}$ -моудлями, то она обратима над $\mathbb{Z}$(Сложнее и правильно всегда понятней чем легче и не совсем правильно.). Поэтому $U^{-1}$ также содержит целые коэффициенты.
\end{proof}

\end{document}